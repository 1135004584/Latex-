\documentclass[UTF8]{ctexart}
\usepackage{indentfirst}
\usepackage{listings}
\usepackage{xcolor}

\lstset{numbers=left,
numberstyle=\tiny,
keywordstyle=\color{blue!70},
commentstyle=\color{red!50!green!50!blue!50},
frame=shadowbox, 
rulesepcolor=\color{red!20!green!20!blue!20},
escapeinside=``,xleftmargin=0em,xrightmargin=0em, 
aboveskip=1em,
language=xml
}
\author{向董董}
\title{代码生成器}
\date{\today}
%缩进2字符

\begin{document}
\setlength{\parindent}{2em}
\maketitle
\newpage
\tableofcontents
\newpage

\part{文件解析}
    \section{行文件}
        \subsection{行文件}

            \par{\textbf{在本代码生成器中,行文件是定义生成的具体内容。}
            行文件固定存放在rows目录下,具体的行文件后缀采用xml文件格式规范,
            所以所有行文件的内容都应该符合xml文件的格式规范。}

            \subsection{行文件的标签}
            \par{行文件目前有5大标签,分别是<rows>、<row>、<append>、<var>、<if>、<for>}
            \\
            \subsubsection{rows标签}
                \par\textbf{rows标签的作用是限定当前所有元素在同一页面排列,意味着rows标签的子标签会被输出到同一个页面。
                }
                \par rows标签有两个属性:\textbf{namespace}和\textbf{name}.
                \\ 其中:
                \\
                \indent namespance表示rows的命名空间,对应着在templet文件中的name.
                \\
                \indent name表示当前rows的名称,以便于在输出文件中进行引用.
                \\\\
                \emph{{例子:}}
                \par\emph{<rows namespace="xxx" name="xxx">...</rows>}

            \subsubsection{row标签}
                \par\textbf{row标签是生成页面的主要元素,表示在生成页面中占据一行},
                row标签只有一个基本属性\textbf{ref},表示当前行引用的model。其中,\emph{在model中定义的属性可以在此引用,并会被注入到Model所被定义的地方。}
                \\
                \\
                \newpage
                \emph{例子:}
                \emph{
                    在model中定义:
                }
\small\begin{lstlisting}
<model name="button">
    <properties>
        <property name="title" default=""></property>
    </properties>

    <source-text>
        <button class="layui-btn layui-btn-primary">
        ${title}
        </button>
    </source-text>
</model>
\end{lstlisting} 

            \emph{在行文件中定义:}
\small\begin{lstlisting}
    <row ref="button" title="登陆">
    </row>
\end{lstlisting}       
\emph{生成结果:}
\small\begin{lstlisting}
    <button class="layui-btn layui-btn-primary">
        登录
    </button>
\end{lstlisting}       



                


\end{document}